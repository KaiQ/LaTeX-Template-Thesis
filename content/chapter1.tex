\chapter{Chapter eins}
bla test.\footnote{http://www.fh-schmalkalden.de}

\lstsethaskell
\begin{lstlisting}[label=listinghaskell,caption=This is Haskell]
module Main where

-- this is a comment
f :: Show a => a -> Int -> String
f x i = show x ++ show i

main :: IO ()
main = do
  putStrLn "Hello World"
  putStrLn $ f 1.2 3
  print $ sum [1..10]
\end{lstlisting}

\lstsetjava
\begin{lstlisting}[label=listingjava,caption=This is \gls{glo:Java}]
// Comment
class Main {
  public static void main(String[] args) {
    System.out.println("Hello World");
  }
}
\end{lstlisting}

\lstsetscala
\begin{lstlisting}[label=listingscala,caption=This is Scala]
// Comment
class Main extends App {
  println("Hello World")
}
\end{lstlisting}

\section{Formeln}

Komplette Referenz zu AMSMath siehe \\
\url{ftp://ftp.ams.org/ams/doc/amsmath/short-math-guide.pdf}

\begin{align}
 \int_{a}^{b} x\,dx
 & = \left.\frac{1}{2} x^2\right\vert_{a}^{b}\\
 & = \frac{1}{2} b^2 - \frac{1}{2} a^2 \\
 \intertext{mit $a=1$ und $b=3$ folgt:}
 \notag
 & = \frac{1}{2} \left(3^2 - 1^2\right)\\
 & = 5
\end{align}

\section{hans wurst}
one more page

\section{hans wurst2}
one more page

\section{hans wurst3}
one more page \footnote{Vgl. \cite{braun:scala}}
